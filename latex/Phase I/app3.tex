\chapter{Generating Random Numbers}
\label{app:C}
In Monte Carlo experiments, random numbers play a central role. Especially 
when modelling real-world phenomena, the quality of random numbers play a 
crucial rule in the reliability of the predictions. While our Monte Carlo 
experiments do not show much evidence of being very dependant on quality 
random number generators, nevertheless they cannot be neglected. So, in 
this Appendix, a brief overview of the random number generation is 
presented. Two major considerations in  random number generation is that 
samples for a  number of distributions can be generated from given samples 
of the uniform distribution, and that the most common naturally occurring 
distribution is the normal or Gaussian distribution. Thus the problem can 
be divided into two areas: 
\begin{enumerate}
\item Generating high quality uniform or normal random numbers, and
\item Using these numbers to generate samples of the required distribution.
\end{enumerate}
True random numbers are based on measuring unpredictable physical 
phenomena, like temperature variations of surroundings, atomic or 
subatomic particle interactions like avalanche effects in diodes, etc. 
These are expensive to obtain, and further are susceptible to both biases 
in the measuring apparatus and any unknown factors which affect the 
measured quantity and cause it to deviate from its natural distribution. 
Such measurements are usually of the normal distribution.

Usually, such random numbers are used to act as initial values or 
\emph{seeds} of pseudorandom number generators, which are computer 
programs that employ multiple mathematical operations to generate 
seemingly unrelated numbers, which are treated as random. In reality these 
numbers are of course not random, since the program itself is 
deterministic. But, if they conform to a number of statistical properties, 
they can be treated as random. The most common (and simplest) of such 
generators are the Linear Congruential generators. However, LC generators 
are not suited for use in Monte Carlo experiments, so they are used to 
provide seeds to higher quality algorithms such as the Mersenne Twister, 
the Blum-Blum-Schub or the lagged Fibonacci generators. Of these, the 
Mersenne Twister algorithm offers a good mix of high quality and 
computational speed, and is often the algorithm of choice. It is, for 
example, the default generator used 
\textsc{MATLAB} \textsuperscript{\textregistered} for generating the 
uniform distribution.

Once we have a high quality generator for the uniform distribution, say 
the Mersenne Twister, one can then use a variety of algorithms to produce 
samples of another distribution, such as the Normal, Expenential, Poisson, Gamma, \citep{Ahrens1993} etc. The algorithms 
will not be described here. Those interested may consult the 
aforementioned articles, and the vast literature on the subject.

Another factor that must be considered is that some algorithms, such as 
the Mersenne Twister or the PG or PT methods \citep{Ahrens1993} generate 
integers. Samples from such generators must be mapped to $\mathbb{R}$ if 
we need to use real numbers, for example during the Monte Carlo searches 
presented here. A common method is to divide the generated integers by 
some constant using floating point arithmetic. This is quite inefficient, 
as only a small fraction ($\sim 7\%$) of the representable floating point 
numbers can be generated this way, and subsequent transformation of these 
values may not yield expected results \citep{Downey2007}. For example, 
applying a log transformation to numbers from an uniform distribution does 
not yield the expect exponential distribution.
