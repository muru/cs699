\chapter{Introduction}
\label{chap:intro}
Second-order dynamic systems with multiple \glspl{dof} are very common,
and a number of methods have been developed to deal with them. Typically, 
these systems take the following form:

\begin{equation}
	\mathbf{M\ddot{r} + C\dot{r} + Kr = f}(t) \label{eqn:basic}
\end{equation}
where \gls{r}, \gls{f} $\in \mathbb{R}^n$ are the vector of physical variables,
and the vector of external forces respectively and  \gls{m}, \gls{k} and \gls{c}
$\in$ \gls{R}. \gls{m} may be considered to be the Mass matrix, \gls{c} the 
Damping matrix and \gls{k} the Stiffness matrix in analogy with the single 
\gls{dof} spring-mass-viscous-damper system given by the equation 
(and the corresponding dimensionless form)
\begin{subequations}
\begin{align}	
	m\ddot{r} + c\dot{r} + kr &=~f(t) \label{eqn:simple}\\
	\ddot{r} + 2\zeta\omega\dot{r} + \omega^2r &=~g(t) \label{eqn:simple_nodim}
\end{align}
\end{subequations}

The multi-degree-of-freedom system in \autoref{eqn:basic} is usually a coupled system, 
and the equations are not dimensionless. Hence solving these equations
analytically is very difficult, but the equations are usually amenable to numerical
solutions. The problem of solving them aside, there is also the vastly more 
difficult problem of controlling such systems. Manipulating a single physical 
variable would affect the entire system due to coupling, and the side-effects may
be counter-productive. Also, the number of physical quantities in consideration 
may be more than the number of linearly independent modes of the system. 
Hence, it is preferable to control the individual \emph{natural modes} of the system, 
instead of the physical displacements. If the natural modes are independent, as
they usually are, the physical displacements are but linear combinations of the 
modes, one may obtain the needed effect by a proper combination of controlling 
forces on the modes. 

To manipulate the modes, one must first obtain the decoupled equations, since
solving these equations yield the modes. Once decoupled, we apply appropriate
modal control forces. The decoupled equations must be obtained through 
\emph{isospectral} transformations, that is, the eigensolution of the coupled
and decoupled systems must be the same. Once the forces have been applied,
inverse transforms have to be used to obtain the physical quantities, which will
be used in the feedback loop. Whether the system is amenable to decoupling has
traditionally been considered to be dependant on the damping matrix \gls{c}.
For the homogeneous system,
\begin{equation}
	\mathbf{M\ddot{r} + C\dot{r} + Kr} = \gls{0} \label{eqn:homogenous}
\end{equation}
\gls{c} may be zero, the system being undamped. This represents the trivial 
case and decoupling this is possible due to results dating back to the $19^{th}$
century.

If the system has \emph{Rayleigh} or \emph{proportional} damping, then 
\gls{c} = $\alpha$\gls{m} + $\beta$\gls{k}. The system may have 
\emph{classical} damping, for which the Caughey-O'Kelly criterion \citep{caughey:583}
applies: 
\[ \gls{c}\gls{m}^{-1}\gls{k} = \gls{k}\gls{m}^{-1}\gls{c} \]
Rayleigh damping is a special case of classical damping. For classically damped
systems, the system maybe decoupled by a simple transformation.

As we shall see, the related eigenvalue problem of the system is quite important in 
modal analysis and control. Therefore, let us first consider mathematics involved.

\section{Mathematical Background}
\label{sec:intro_math}
% Systems of the form \autoref{eqn:basic} are closely related to eigenvalue problems,
% which vary in form depending mostly on \gls{c}.  
% However, the criteria for classical damping has been called into question, 
% a point to which we shall return later.
Aside from the form presented here, one may also choose to go about using decoupling
via the state space form of \autoref{eqn:homogenous}. The state space form
presents its own eigenvalue problem. However the decoupled equations obtained thus 
are different from the natural modes. The natural modes would be the equivalents of 
the dimensionless form (\autoref{eqn:simple_nodim}). To obtain these, one must diagonalize 
the system matrices and not the state-space matrices.

\subsection{No Damping $ (\mathbf{C = 0})$}
\autoref{eqn:homogenous} now reduces to
\begin{equation}
	\mathbf{M\ddot{r} + Kr = 0} \label{eqn:no_damp}
\end{equation}
the corresponding generalized eigenvalue problems being:
\begin{subequations}
\label{eqn:eigen_problem}
\begin{align}
	\mathbf{Kx} =  &-\gls{l}\mathbf{Mx}, \label{eqn:eigen_problem_right}\\
	\gls{y}\gls{trans}\gls{k} =  &-\gls{l}\gls{y}^T\gls{m}, \label{eqn:eigen_problem_left}
\end{align}
\end{subequations}
The eigenvalue problem, when written as a matrix polynomial in \gls{l},
is called a pencil, and is termed linear, quadratic, etc. depending 
on the order of the polynomial. The matrix pencil \gls{lp} in this case is:
\begin{equation}
	\gls{lp} := \gls{m}\gls{l} + \gls{k} = 0 \label{eqn:linear_pencil}
\end{equation}
The eigensolution is:
\begin{align}
  \gls{L} = 
  	&\begin{bmatrix}
		\gls{l}_1 & & \\
		 & \ddots & \\
		 & & \gls{l}_n \\
	\end{bmatrix} \in \gls{C}, 
	\gls{l}_1 \geq \ldots \geq \gls{l}_n\\
	\gls{X} = 
 	&\begin{bmatrix}
		\gls{x}[_1] & \dots & \gls{x}[_n] 
	\end{bmatrix} \in \mathbb{C}^{n \times n}\\
	\gls{y} = 
 	&\begin{bmatrix}
		\gls{y}[_1] & \dots & \gls{y}_n
	\end{bmatrix} \in \mathbb{C}^{n \times n}
\end{align}
Now, if the matrices \gls{m} and \gls{k} are symmetric, then we
can have \gls{X} $=$ \gls{Y} but we shall consider the more general,
asymmetric,	case. For this introduction let us make an assumption, that the
system is not defective\footnote{A system with $n$ eigenvalues is defective 
iff it does not have $n$ linearly independent eigenvectors.}\glsadd{det}:
\begin{equation}
	\left\vert{\gls{X}}\right\vert \neq 0 \neq \left\vert{\gls{Y}}\right\vert \label{eqn:asptn_defective}
\end{equation}
Traditionally, from this point on, one would usually encounter similarity
transformations \citep{watkins2010fundamentals}. Thus:
\begin{align}
 	\gls{md} &= \gls{X}^{-1}\gls{m}\gls{X}\\
 	\gls{kd} &= \gls{X}^{-1}\gls{k}\gls{X}
\end{align}
If the matrices were special in some way, say, if they were \emph{normal}\footnote{%
Matrix $\mathbf{A}$ is said to be normal iff 
$\mathbf{AA}\gls{ctrans} = \mathbf{A}\gls{ctrans}\mathbf{A}$.}, 
one would expect some unitary similarity transformations\footnote{If $\mathbf{U} \in \gls{C}$ 
is a unitary matrix, $\mathbf{U}\gls{ctrans}\mathbf{U} = \mathbf{UU}\gls{ctrans} = \gls{I}$. 
Orthogonal matrices are real unitary matrices.}. 
Or, if they were symmetric, which is often the case, 
orthogonal transformations ($\gls{X}\gls{trans}\gls{X} = \gls{I}$):
\begin{align}	
 	\gls{md} &= \gls{X}^{T}\gls{m}\gls{X}\\
 	\gls{kd} &= \gls{X}^{T}\gls{k}\gls{X}
\end{align}

However, let us depart from the traditional way, and inspect the use of both left
and right eigenvectors. It turns out that we can diagonalize \gls{m} and 
\gls{k} by the \emph{isospectral} transformations \citep{Chu200896}
\begin{align}
 	\gls{md} &= \gls{Y}^T\gls{m}\gls{X}\\
 	\gls{kd} &= \gls{Y}^T\gls{k}\gls{X}
\end{align}
The term “isospectral” is used in the strong sense that the eigenvalues and 
all their partial multiplicities are common to isospectral systems. 
Note that all similarity transforms are isospectral, but the converse is not 
necessarily true. Now, if the eigenvectors given by \gls{X} and \gls{Y}
are normalized, we have,
\begin{subequations}
\label{eqn:MK_diag}
\begin{align}
	\gls{I} &= \gls{Y}^T\gls{m}\gls{X}\\
	\gls{L} &= \gls{Y}^T\gls{k}\gls{X}
\end{align}
\end{subequations}
yielding the decoupled equations
\[ \mathbf{\ddot{q} + \gls{l}{q} = 0, \quad q = Xr}\]
which represent the individual modes of the system.

\subsection{Classical Damping}
For classically damped systems, the eigenvalue problem remains unchanged. Further,
the transformations that diagonalize \gls{m} and \gls{k} (\autoref{eqn:MK_diag}) 
also diagonalize \gls{c}. This is trivially so for Rayleigh
damping. The proof is not of essence to our task, so it shall not be stated here.
However those interested may consult \citet{Inman15012001} for a short proof
that the Caughey-O'Kelly criterion is the necessary and sufficient condition
for this.

\subsection{General damping}
In the event that damping is not classical, the transformations are not 
guaranteed to simultaneously diagonalize all three matrices.  In 
particular, gyroscopic systems have skew-symmetric damping matrices. The 
eigenvalue problem is also different. It is no longer linear, but quadratic:
\begin{subequations}
\label{eqn:quadratic_problem}
\begin{align}
	(\gls{m}\gls{l}^{2} + \gls{c}\gls{l} + \gls{k})\mathbf{x} &= 0 \label{eqn:quadratic_pencil_r} \\
	\gls{y}^T(\gls{m}\gls{l}^{2} + \gls{c}\gls{l} + \gls{k}) &= 0 \label{eqn:quadratic_pencil_l}\\
	\gls{qp} := \gls{m}\gls{l}^{2} + \gls{c}\gls{l} + \gls{k} &= 0 \label{eqn:quadratic_pencil}
\end{align}
\end{subequations}	
This eigenvalue problem, is no longer $n$-dimensional, but $2n$-dimensional,
since there are $2n$ eigenvalues which satisfy the quadratic pencil. Hence the
computational complexity of the problem becomes considerably higher.

\subsection{Defective systems}
We made an assumption that the system is not defective (\autoref{eqn:asptn_defective}).
Defective matrices are rare enough that one is justified in making this assumption. 
Defective matrices may not be fully diagonalized, but they can be converted into 
the Jordan normal form:
\begin{align}
 \gls{L} &= \begin{bmatrix}
	\gls{L}_1 & & & \\
	 & \gls{L}_2 & & \\
	 & & \ddots & \\
	 & & & \gls{L}_n	
\end{bmatrix} \\
\gls{L}_i &= \begin{bmatrix}
	\gls{l}_i & 1 & & \\
 	 & \gls{l}_i & \ddots & \\
 	 & & \ddots & 1 \\
 	 & & & \gls{l}_i
\end{bmatrix}
\end{align}
The \gls{L}[$_i$] are called Jordan blocks.
The number of Jordan blocks for an eigenvalue is its geometric multiplicity,
and the sum of the sizes of the blocks is its algebraic multiplicity. The 
matrix is diagonalizable iff both multiplicities are same for all eigenvalues.
If this is not the case, then the modes involving a common eigenvalue
might be coupled with each other, as they are not independent. Defective 
systems are very sensitive to perturbation, and even systems close to
defective matrices suffer from this problem. Therefore an error in 
computation, due to round-off error, or truncation, can change the 
characteristics of the problem. 
%For example, purely imaginary eigenvalues may become complex.

\section{Modal Control in General}
\label{intro_modal_general}
Modal Control enjoyed its heyday in the 1980s and 1990s. Then it fell out of favour, 
due to problems involving uncontrolled modes and spillover, which we shall discuss in 
the next chapter. Another major drawback was the requirement of a large number of 
sensors to monitor all the modes. Further, \gls{imsc} and other methods are not suited 
for problems involving gyroscopic systems. \citet{Meirovitch19791} has presented 
improvements in \gls{imsc} which handle gyroscopic systems with slight damping,
a restriction which we seek to remove. The Caughey-O'Kelly criterion has been called 
into question, and qualifications added \citep{A2003741}. Thus, there is a need for 
general decoupling transformations.

However, not only do we need general decoupling transformations, we need \emph{real-valued}
transformations. Complex transforms, while mathematically convenient to obtain,
add a layer of complexity in practical computation. The Caughey-O'Kelly condition,
insofar as it holds, do provide for such transformations, from results known to
mathematicians from the times of Weierstrass. Another requirement is that the
transformations be \emph{stable}. Stability can be considered in two contexts -- 
numerical stability and stability of the eigensystem with respect to perturbation.
If the transforms are not numerically stable, errors in the system will be
magnified, and attempts at control will prove expensive and ineffectual. Since
modal control aims to modify the modes, i.e., the eigenvalues and eigenvectors,
unstable perturbations of the eigensystem would prove counter-productive.

\subsection{True Modal Control}
True Modal Control aims to use such general decoupling transformations to completely
decouple systems with arbitrary damping. Then each mode shall be controlled 
independently, such that the overall effect is to coerce the physical variables
to the desired state. The transformations involved would be real-valued. Thus,
True Modal Control will address the problems of spillover, since all modes will
be modelled. Further, characteristics of the system such as gyroscopic damping,
which often proves stabilizing, would be retained, and put to use in an
advantageous manner.

There have been recent advances in this field by \citet{GARVEY2002885,GARVEY2002911}
and \citet{Chu200896,Chu2008112}. This project aims to build on and improve 
their results, both in a mathematical sense and in a computational sense. One case
that is side-stepped by current strategies is a zero eigenvalue, or, equivalently,
a system with rigid-body modes. Accommodating this case and that of defective systems
is another goal of this project. We shall discuss these and the various modal 
control strategies in the next chapter. %These methods, as well as IMSC, convert the 
%$n$-\gls{dof} problem to $2n$-\gls{dof}.

