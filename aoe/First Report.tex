\documentclass[a4paper,12pt]{article}
%\usepackage{footnotebackref}
\usepackage{fullpage}
\usepackage{program}
\hyphenpenalty=750
\usepackage[hyperfootnotes=true]{hyperref}

% Title Page
\title{AI Scripting for \\Microsoft{\small \texttrademark} Age of Empires II{\small \texttrademark}}
\author{Balajiganapathi Senthilnathan\\Murukesh Mohanan\\Nilang Shah}

\begin{document}
\maketitle

\section*{Aim:}
To write an AI script that will defeat the standard AI of 
Age of Empires II: The Conquerors 1.0c (AoC) using principles learned in 
CS 621: Artificial Intelligence, and the AI scripting mechanism provided 
by Microsoft for AoC.

\section*{Introduction}
Microsoft{\small \texttrademark} Age of Empires II: 
Age of Kings{\small \texttrademark} (AoK) is one of the most popular 
Real-Time Strategy games of our time. Originally released in 1999 and 
followed by an expansion pack called Age of Empires II: 
The Conquerors{\small \texttrademark} in 2000. It has developed an 
extensive, thriving community centred on developing custom maps, 
scenarios, campaigns and AIs.\footnote{The 
\href{http://www.aok.heavengames.com/}{Age of Kings Heaven} is perhaps 
the largest of these.} The immense popularity of 
AoC has spawned a number of modified maps, open-source clones, etc.\footnote{A popular mod called  
\href{http://userpatch.aiscripters.net/}{The User Patch}
considerably enhances the game, in graphics capability, AI  
configurability, and a lot more. However, at least initially, 
we won't be using it.}


AoC has 18 different civlizations, each with its own unique technologies 
and units. The Tech Tree of common technologies spans four distinct ages, 
representing real historic periods.  The wide variety available in units, 
technologies, formations, terrain and other parts of the game make it  
very complex game in terms of AI.

The gameplay involves a player controlling some units and buildings to 
gather resources, research technologies and build more powerful units and 
buildings. The end target varies from utterly destroying the enemy 
players' units and buildings (Conquest) to guarding a position or an 
object for a certain period of time (King of the Hill) to killing a 
specific unit of the enemy player (Regicide). 

The multiplayer mode of the game involves humans and computer-controlled 
players teaming up in various combinations. Multiple humans can control 
the same in-game player, or humans can co-operate with the computer and 
control a single player.

\subsection*{Economy}
Civilian units, called ``villagers'', are used to gather resources. 
Resources can be used to train units, construct buildings, and research 
technologies, among other things; for example, players can research 
better armour for infantry units. The game offers four types of resources: 
food, wood, gold, and stone. Food is obtained by hunting animals, 
gathering berries, harvesting livestock, farming, and shore fishing and 
fishing from boats. Wood is gathered by chopping down trees. Gold is 
obtained from either gold mines, trade or collecting relics in a 
monastery, and stone is collected from stone mines. Villagers require 
checkpoints, typically depository buildings (Town Center, mining camps, 
mills, and lumber yards), where they can store gathered resources.

Every player has a limit to the number of units they can create — a 
population limit — but may not immediately use the entire potential 
population. The population capacity, which can be capped at anywhere 
between 75 - 200 in intervals of 25, is based on the number of houses, 
Castles, or Town Centers -- the main building in a player's town -- which 
have been built.

\subsection*{Strategy and Logistics}
The land military units are broadly categorized into infantry, cavalry, 
archers and siege weapons. Most types have a corresponding counter-unit, 
which is specialized against the original. There are units which blur the 
lines - such as cavalry archers or camel-riders. Further, the individual 
civilizations have their own strengths and weakness when it comes to the
various military units. Franks (equivalent to modern-day French) produce 
stronger cavalry than most other civilizations, whereas the Teutons 
produce particularly powerful infantry. The Celts feature infantry 
using throwing axes, thus giving them ranged abilities without the typical 
weakness of archers. And so on for the other civilizations.

Strategic decisions are of immense importance in a Real-Time Strategy game 
like AoC. The layout of the terrain, availability of resources, locations 
of allies, strengths and weaknesses of their civilizations, all go into 
deciding a good strategy, which acts as an overall guide, or framework,  
using which the individual players make appropriate tactical decisions.
We must also consider the logistics involved: 
\begin{itemize}
	\item Movement of armies:
	\begin{itemize}
		\item Over land:	Slower, delicate units such as siege weapons and 
		monks need to be protected. Exploit the speed of mounted units to 
		ward off attacks on the main army.
		\item Over sea:	Transport ships have limited capacity and units 
		in ships cannot defend themselves. If the enemy is across a body 
		of water, one must set up a base on enemy territory. 
	\end{itemize}
	\item Replenishment of armies after battles and skirmishes:
	\begin{itemize}
		\item Maintaining a level of resources, to allow for 
		quick replenishment.
		\item Ensuring proper timing of arrival of reinforcements.
	\end{itemize}
	\item Retreating under fire while minimizing losses.
\end{itemize}

\section*{Scripting}
Scripting in AoC is based on a proprietary system provided by Microsoft. 
A vaguely Lisp-like language is used throughout. The essential form of a 
command in a script looks like this:
\begin{program}
( 
	\emph{set of conditions} =>	\emph{set of actions}
)
\end{program}
While the basic form of the language might look simple, the number of 
conditions and actions available is immense. Aside from these conditions 
and actions, there are also hundreds of game-play parameters called 
strategic numbers. Changing some of these numbers can drastically alter 
the course of game.

Since the AI engine is able to run through the script at speeds well above 
the average human's reactions, AI players face some limitations. For 
example, only one player might be started on constructing a building. The 
default strategic numbers that govern an AI player's military units are 
also different.

Finally, unlike in the original Age of Empires, the standard AI engine in 
AoC can, and (at higher difficulty levels) does, cheat.\footnote{\href{http://web.archive.org/web/20081013144438/http://www.microsoft.com/games/empires/behind_dave.htm}
{Archived copy of AoC AI specialist Dave Pottinger's interview.}} 
This is done, for example, by the standard AI giving itself free resources 
on occasion. But an AI script cannot benefit from this ``feature''. All 
these issues must be taken into account when writing the script.

\section*{Conclusion}
Implementing the AI script can be broadly classified into three sections:
\begin{enumerate}
	\item Balancing the economy - maintaining levels of production, 
	researching technology and advancing through the ages, maintaining 
	population (both available and used), etc.
	\item Managing the military - production of balanced armies, timing 
	the attacks, replenishing units and so on.
	\item Fine-tuning the strategy - Adding rules for each civilization's 
	strengths and weaknesses, variations based on the terrain and game 
	style and so on. 
	Thus an overall strategy coordinates the economy and military.
\end{enumerate}

Writing a strong AI script requires both focusing the details and, at the 
same time, keeping an eye on the big picture. We believe that the innate 
duality of the details - economic and military - combined with the need 
for attention to the overall scheme of things make this project perfect 
for three people.

\end{document}          
