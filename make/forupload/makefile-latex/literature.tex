\begin{multicols}{2}
\section{LitRev}
%\chapter{Literature Review}
\label{chap:lit_rev}
Theories on Modal Control have been around for around half a century, initially 
suggested by \citet{Rosen1962}. \citet{Simon1968316} expanded on this work. 
Modal control theories emerged independently in two fields - chemical engineering,
and structural dynamics. The concepts which evolved in structural engineering 
were formulated into \gls{imsc} by Prof Meirovitch and his team in the 1980s, and 
this work has been summarized in \citet{meirovitch1990dynamics}.

\section{Independent Modal Space Control}
\label{sec:lit_imsc}
In \gls{imsc}, one decouples the equations \eqref{eqn:basic} by unitary 
similarity transforms. In the event that \gls{c} is not diagonalized,
we ignore the off-diagonal terms. Further, the control forces are only applied 
to the lower modes, since the higher modes are difficult to physically monitor 
or control. Thus, IMSC is most effective in situations where only a few critical 
modes need be studied and controlled. However, this strategy exposes itself by 
two grievous flaws in the general case. Firstly, when the off-diagonal terms are 
of comparable magnitude to the diagonal terms, ignoring them is a very poor 
approximation. In rotating systems, for example, the skew-symmetry of the 
damping matrix is due to the gyroscopic nature of the system, and ignoring
off-diagonal terms here is ignoring the gyroscopic effects themselves -- and
the very essence of the problem. It may happen that the decoupled equations 
thus obtained have quite different eigenvalues from the original. There is 
also the problem of deciding how many modes one must model, how many to leave 
unmodelled, and how many of the modelled modes be controlled. This question is 
further complicated by the different needs of open and closed loop control 
systems. What may be satisfactory for one may not be so for the other.

Secondly, trying to control only the lower modes raises the problem of spill-over.
Spill-over refers to applying a control law designed around a limited
model of a system, to the entire system and inadvertently exciting that part of the
model discarded (called residual modes in a modal model of a structure). Since the 
higher modes are not in the model, the effect will be different from that predicted
by the control design. In some cases, the modal control forces may significantly 
increase the contributions of uncontrolled modes to the vibration of the system, 
especially in cases like flexible structures, in which the contributions of its 
higher modes cannot be ignored. Thus while using a modal model one must be careful
not to excite the unused modes with the control system designed to improve the 
structure's response. Otherwise, control energy will `spill over' into the residual 
modes and excite these modes, spoiling the response and desired performance is not 
achieved. 

As mentioned earlier, IMSC converts the $n$-DOF second-order system to a $2n$-DOF
first-order system. That is, the system is linearised. Useful characteristics of 
the system matrices such as symmetry and definiteness are lost. A \gls{mimsc} 
theory as presented by \citet{Fang2003421} seeks to address shortcomings of 
\gls{imsc}, such as spill over.
\end{multicols}
\section{\glsentryfirstplural{spe}}
\label{sec:SPT}
The quadratic pencil \eqref{eqn:quadratic_pencil} can be linearized to different forms.
\marginnote{Undefined references are to sections that I couldn't be bothered to copy.}
Consider the following matrices, called \glspl{lam}:
\begin{equation}
	\label{eqn:lam}
	\gls{a} = \begin{bmatrix}
	\gls{k} & \gls{0}\\
	\gls{0} & -\gls{m}
	\end{bmatrix} 
	\quad\mathbf{B} = \begin{bmatrix}
	\gls{c} & \gls{m}\\
	\gls{m} & \gls{0}
	\end{bmatrix}  
	\quad\mathbf{D} = \begin{bmatrix}
	\gls{0} & \gls{k}\\
	\gls{k} & \gls{c}
	\end{bmatrix}
\end{equation}
and the \emph{companion matrices}, often seen in the state-space form of system equations:
\begin{equation}
	\gls{c}_R = \begin{bmatrix}
	\gls{0} & \gls{I}\\
	\mathbf{-M}^{-1}\gls{k} & \mathbf{-M}^{-1}\gls{c}
	\end{bmatrix} 
	\qquad \gls{c}_L = \begin{bmatrix}
	\gls{0} & \gls{k}\gls{m}^{-1}\\
	\gls{I} & \gls{c}\gls{m}^{-1}
	\end{bmatrix}
\end{equation}
If we let $\mathbf{p} = \mathbf{\dot{r}}$ and $\mathbf{P} = \mathbf{\dot{f}}$, 
then we can write equation \eqref{eqn:basic} in any of the following state-space 
forms:
\begin{align}
\label{eqn:stsp_g}
	\mathbf{D}
	\begin{bmatrix}
		\mathbf{r}\\
		\mathbf{p}
	\end{bmatrix} - \gls{a}
	\begin{bmatrix}
		\mathbf{\dot{r}}\\
		\mathbf{\dot{p}}
	\end{bmatrix} &= 
	\begin{bmatrix}
		\gls{0}\\
		\gls{I}
	\end{bmatrix}\mathbf{f}\\
	\gls{a}
	\begin{bmatrix}
		\mathbf{r}\\
		\mathbf{p}
	\end{bmatrix} + \mathbf{B}
	\begin{bmatrix}
		\mathbf{\dot{r}}\\
		\mathbf{\dot{p}}
	\end{bmatrix} &= 
	\begin{bmatrix}
		\gls{I}\\
		\gls{0}
	\end{bmatrix}\mathbf{f}\\
	\mathbf{D}
	\begin{bmatrix}
		\mathbf{r}\\
		\mathbf{p}
	\end{bmatrix} + \mathbf{B}\begin{bmatrix}
		\mathbf{\dot{r}}\\
		\mathbf{\dot{p}}
	\end{bmatrix} &= 
	\begin{bmatrix}
		\mathbf{P}\\
		\mathbf{f}
	\end{bmatrix}
\end{align}
Now, two non-singular matrices $T_L$ and $T_R \in \mathbb{C}^{2n \times 2n},$ 
can be considered to be \glspl{spe}\footnote{As 
used by \citet{GARVEY2002885,GARVEY2002911}, there is a restriction on the 
above definition: that $\gls{m}_0$ not be singular. We shall modify this 
restriction: $\gls{m}_0 \neq \gls{0}$. The reason shall be apparent later.},
if the isospectral transforms $\gls{a}_0 = \mathbf{T}_L\mathbf{A_0T}_R, 
\mathbf{B}_0 = \mathbf{T}_L\mathbf{B_0T}_R$ and $\mathbf{D}_0 =
\mathbf{T}_L\mathbf{D_0T}_R$, retain their block-structure, that is, if
\begin{align}
	\mathbf{A_0} = \begin{bmatrix}
	\gls{k}_0 & \gls{0}\\
	\gls{0} & -\gls{m}_0
	\end{bmatrix} \quad
	\mathbf{B_0} = \begin{bmatrix}
	\gls{c}_0 & \gls{m}_0\\
	\gls{m}_0 & \gls{0}
	\end{bmatrix}\quad
	\mathbf{D}_0 = \begin{bmatrix}
	\gls{0} & \gls{k}_0\\
	\gls{k}_0 & \gls{c}_0
	\end{bmatrix}
\end{align}
Utilisation of the \glspl{spe} permits the diagonalization of the system mass, 
damping and stiffness matrices for non-classically damped systems, as shown by 
\citet{GARVEY2002885,GARVEY2002911}. A modal control method is presented by 
\citet{Houlston2007}, which exploits this diagonalization. The method 
introduces independent modal control in which a separate modal controller is 
designed in modal space for each individual mode or pair of modes. The theory 
of decoupling used is presented in a concise form in \citet{Friswell2001}. We 
shall present the basic equations here. 

Except for defective systems, it is possible to find matrices
$\mathbf{W}_L,$ $\mathbf{X}_{L},$ $\mathbf{Y}_{L},$ $\mathbf{Z}_L$ and $\mathbf{W}_R,$ 
$\mathbf{X}_R, $ $\mathbf{Y}_R, $ $\mathbf{Z}_R$ in \gls{R}
such that
\begin{align}
	\begin{bmatrix}
		\mathbf{W}_L & \mathbf{X}_{L}\\
		\mathbf{Y}_{L} & \mathbf{Z}_L
	\end{bmatrix}^T
	\begin{bmatrix}
		\gls{0} & \phantom{-}\gls{k}\\
		\gls{k} & \phantom{-}\gls{c}
	\end{bmatrix}
	\begin{bmatrix}
		\mathbf{W}_R & \mathbf{X}_R\\
		\mathbf{Y}_R & \mathbf{Z}_R
	\end{bmatrix} &= 
	\begin{bmatrix}
		\gls{0} & \mathbf{\Omega}^2\\
		\mathbf{\Omega}^2 & (2\mathbf{\zeta\Omega})
	\end{bmatrix}\\	
	\begin{bmatrix}
		\mathbf{W}_L & \mathbf{X}_L\\
		\mathbf{Y}_L & \mathbf{Z}_L
	\end{bmatrix}^T
	\begin{bmatrix}
		\gls{k} & \phantom{-}\gls{0}\\
		\gls{0} & -\gls{m}
	\end{bmatrix}
	\begin{bmatrix}
		\mathbf{W}_R & \mathbf{X}_R\\
		\mathbf{Y}_R & \mathbf{Z}_R
	\end{bmatrix} &= 
	\begin{bmatrix}
		\mathbf{\Omega}^2 & \gls{0} \\
		\gls{0}  & -\gls{I} 
	\end{bmatrix}\\	
	\begin{bmatrix}
		\mathbf{W}_L & \mathbf{X}_{L}\\
		\mathbf{Y}_L & \mathbf{Z}_L
	\end{bmatrix}^T
	\begin{bmatrix}
		\gls{c} & \phantom{-}\gls{m}\\
		\gls{m} & \phantom{-}\gls{0}
	\end{bmatrix}
	\begin{bmatrix}
		\mathbf{W}_R & \mathbf{X}_R\\
		\mathbf{Y}_R & \mathbf{Z}_R
	\end{bmatrix} &= 
	\begin{bmatrix}
		(2\mathbf{\zeta\Omega}) & \gls{I}\\
		\gls{I} & \gls{0}
	\end{bmatrix}	
\end{align}
where, both $\mathbf{\Omega}^2$ and $2\mathbf{\zeta\Omega}$ are real-valued
and diagonal. Now,
\begin{align}
\mathbf{Y}_{L} = -\mathbf{X}_{L}\mathbf{\Omega}^2, \quad
\mathbf{W}_L = \mathbf{Z}_L + \mathbf{X}_{L}(2\mathbf{\zeta\Omega})
\end{align}
A similar relation holds for the right matrices. Consider the eigenvalue 
problem obtained from the first of the state-space forms in \eqref{eqn:stsp_g}. 
Let \gls{X} be the matrix composed of right eigenvectors and \gls{Y}
be the matrix composed of left eigenvectors. Considering a block structure
using matrices similar to $\mathbf{W}_L,$ $\mathbf{X}_L,$ $\mathbf{Y}_L,$, etc.,
let $\mathbf{\Phi}_{L1},$ $\mathbf{\Phi}_{L2},$ $\mathbf{\Theta}_{L2},
$ $\mathbf{\Theta}_{L2}$ form the block structure of \gls{X}, with a similar
structure for \gls{Y}. Then, rewriting \eqref{eqn:MK_diag} using the block
structure:
\begin{align}
\label{eqn:stsp_eig}
	\begin{bmatrix}
		\mathbf{\Phi}_{L1} & \mathbf{\Phi}_{L2}\\
		\mathbf{\Theta}_{L1} & \mathbf{\Theta}_{L2}
	\end{bmatrix}^T
	\begin{bmatrix}
		\gls{0} & \phantom{-}\gls{k}\\
		\gls{k} & \phantom{-}\gls{c}
	\end{bmatrix}
	\begin{bmatrix}
		\mathbf{\Phi}_{R1} & \mathbf{\Phi}_{R2}\\
		\mathbf{\Theta}_{R1} & \mathbf{\Theta}_{R2}
	\end{bmatrix} &= 
	\begin{bmatrix}
		\gls{L}_1 & \gls{0}\\
		\gls{0} & \gls{L}_2
	\end{bmatrix}\\	
	\begin{bmatrix}
		\mathbf{\Phi}_{L1} & \mathbf{\Phi}_{L2}\\
		\mathbf{\Theta}_{L1} & \mathbf{\Theta}_{L2}
	\end{bmatrix}^T
	\begin{bmatrix}
		\gls{k} & \phantom{-}\gls{0}\\
		\gls{0} & -\gls{m}
	\end{bmatrix}
	\begin{bmatrix}
		\mathbf{\Phi}_{R1} & \mathbf{\Phi}_{R2}\\
		\mathbf{\Theta}_{R1} & \mathbf{\Theta}_{R2}
	\end{bmatrix} &= 
	\begin{bmatrix}
		\gls{I} & \gls{0} \\
		\gls{0}  & \gls{I} 
	\end{bmatrix}
\end{align}
Of the $2n$ eigenvalues obtained, counting multiplicity, let $2p \leq 2n$
be the number of complex eigenvalues with $2q$ real eigenvalues. The matrix 
which we now define will also be used later, since a modified version plays 
the same role in the theory presented by \citet{Chu200896}.
\begin{align}
	\mathbf{J} &= \begin{bmatrix}
	\frac{1}{\sqrt{2}}\gls{I}_p & \gls{0} & \frac{-\iota}{\sqrt{2}}\gls{I}_p & \gls{0} \\
	\gls{0} & \gls{I}_q & \gls{0} & \gls{0} \\
	\frac{1}{\sqrt{2}}\gls{I}_p & \gls{0} & \frac{\iota}{\sqrt{2}}\gls{I}_p & \gls{0} \\
	\gls{0} & \gls{0} & \gls{0} & \iota\gls{I}_q \\
	\end{bmatrix}\\
\mathbf{J}^T\mathbf{\Lambda}\mathbf{J} &= \begin{bmatrix}
		\mathbf{\Lambda}_x & \mathbf{\Lambda}_y\\
		\mathbf{\Lambda}_y & \mathbf{\Lambda}_z
	\end{bmatrix}
\end{align}
Now select a diagonal matrix \gls{g} $\in$ \gls{C}, such that
\begin{align}
	\begin{bmatrix}
		\cosh{\gls{g}} & \sinh{\gls{g}}\\
		\sinh{\gls{g}} & \cosh{\gls{g}}\\
	\end{bmatrix}
	\begin{bmatrix}
		\mathbf{\Lambda}_x & \mathbf{\Lambda}_y\\
		\mathbf{\Lambda}_y & \mathbf{\Lambda}_z
	\end{bmatrix}
	\begin{bmatrix}
		\cosh{\gls{g}} & \sinh{\gls{g}}\\
		\sinh{\gls{g}} & \cosh{\gls{g}}\\
	\end{bmatrix} = 
	\begin{bmatrix}
		\gls{0} & \mathbf{\Omega}^2\\
		\mathbf{\Omega}^2 & (2\mathbf{\zeta\Omega})
	\end{bmatrix}
\end{align}
Then the following relation (with its equivalent for the right eigenvectors) 
give us the required diagonalizing \glspl{spe}:
\begin{align}
	\begin{bmatrix}
		\mathbf{W}_L & \mathbf{X}_L\\
		\mathbf{Y}_L & \mathbf{Z}_L
	\end{bmatrix} &= 
	\begin{bmatrix}
		\mathbf{\Phi}_{L1} & \mathbf{\Phi}_{L2}\\
		\mathbf{\Theta}_{L1} & \mathbf{\Theta}_{L2}
	\end{bmatrix}\mathbf{J}
	\begin{bmatrix}
		\cosh{\gls{g}} & \sinh{\gls{g}}\\
		\sinh{\gls{g}} & \cosh{\gls{g}}\\
	\end{bmatrix}
	\begin{bmatrix}
		\mathbf{\Omega} & \gls{0} \\
		\gls{0}  & \gls{I} 
	\end{bmatrix}
\end{align}

There are two points to consider in this rather straightforward relation.
First is obtaining an appropriate \gls{g}, which is where the bulk of the 
computation lies. Secondly, there is no simple way whereby this method 
could be adapted for systems with eigenvalue zero. \citet{Chu200896}
derived a modified version of this relation, which, as we shall show in the
next chapter, is relatively easily adapted for this case. The transformation 
derived above is not unique, and can be arrived via a different route, albeit 
sharing a few steps. Another route to decoupling transformations is also 
discussed in \citet{Friswell2001}, which uses Clifford algebra, specifically,
\gls{cl}. We shall not discuss this method, since Clifford algebras are 
beyond the scope of this work.
